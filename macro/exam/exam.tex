\documentclass{article}
\usepackage{graphicx} % Images
\usepackage{amsmath,amsthm} % Math
\usepackage{wasysym,MnSymbol} % Greek alphabets
\usepackage{mathrsfs,amsfonts,calrsfs} % Math fonts
\usepackage{newtxtext}
\usepackage{geometry} % Formatting
\usepackage[dvipsnames,svgnames]{xcolor}
\usepackage[strict]{changepage}
\usepackage{framed}
\usepackage{cancel}
\usepackage{autobreak}
\usepackage{hyperref}

\graphicspath{ {./images/} }

\providecommand{\tightlist}{
  \setlength{\itemsep}{0pt}
  \setlength{\parskip}{0pt}}

\newcommand*\sepline{%
  \begin{center}
    \rule[1ex]{.5\textwidth}{.5pt}
  \end{center}}


\setlength{\parindent}{0pt}% cancel indentation before every line

% right cases
\newenvironment{rcases}
  {\left.\begin{aligned}}
  {\end{aligned}\right\rbrace}

% Blue block
\definecolor{blueshade}{rgb}{0.95,0.95,1} % blue block color
\newenvironment{blueblock}{
\def\FrameCommand{
  \hspace{1pt}
    {\color{DarkBlue}
    \vrule width 2pt}
    {\color{blueshade}
    \vrule width 4pt}
  \colorbox{blueshade}
}
\MakeFramed{
  \advance
  \hsize-
  \width
  \FrameRestore}
\noindent\hspace{-4.55pt}% disable indenting first paragraph
\begin{adjustwidth}{}{7pt}
\vspace{2pt}\vspace{2pt}
}
{\vspace{2pt}\end{adjustwidth}\endMakeFramed}


% Green block
\definecolor{greenshade}{rgb}{0.90,0.99,0.91} % green block
\newenvironment{greenblock}{%
\def\FrameCommand{%
  \hspace{1pt}%
    {\color{Green}%
    \vrule width 2pt}%
    {\color{greenshade}%
    \vrule width 4pt}%
  \colorbox{greenshade}%
}%
\MakeFramed{%
  \advance%
  \hsize-%
  \width%
  \FrameRestore}%
\noindent\hspace{-4.55pt}% disable indenting first paragraph
\begin{adjustwidth}{}{7pt}%
\vspace{2pt}\vspace{2pt}%
}
{%
\vspace{2pt}\end{adjustwidth}\endMakeFramed%
}


% Red block
\definecolor{redshade}{rgb}{1.00,0.90,0.90} % red block
\newenvironment{redblock}{
\def\FrameCommand{
  \hspace{1pt}
    {\color{LightCoral}
    \vrule width 2pt}
    {\color{redshade}
    \vrule width 4pt}
  \colorbox{redshade}
}
\MakeFramed{
  \advance
  \hsize-
  \width
  \FrameRestore}
\noindent\hspace{-4.55pt}% disable indenting first paragraph
\begin{adjustwidth}{}{7pt}
\vspace{2pt}\vspace{2pt}
}
{\vspace{2pt}\end{adjustwidth}\endMakeFramed}


\newtheorem{question}{Question}
\newtheorem{note}{Note}

\title{Notes on Advanced Macroeconomics, Edition for Final Exam}
\author{Victor Li}
\date{June Sememster, 2024}

\begin{document}

\maketitle




















\section{jian da ti, 15'x2}

\subsection{RBC model (6 eqs) vs classical model (7 eqs), match similar eqs and differ}

\includegraphics[width=\textwidth]{images/Classical_vs_RBC.jpg}

In conclusion, classical model assumes a world where market always clears based on simple aggregated relationships; RBC model, using micro foundation and incorporating technology shocks, still portraits a perfectly competitive world but with mechanism rendering market-unclearing possible. The key difference is that RBC model, originated from Ramsey model, internalizes consumption and saving decisions. Therefore RBC consists of six equation, where as classical consists of seven.














\subsection{?}














\newpage
\section{jisuan tuidao ti, 20'x2}







\subsection{second way to deduct NKPC, three equations, acceptable price setting eq}

Now sticky PAE\footnote{See more in Calvo, 1983} is
\begin{align}
& \color{red} \hat p_t^\triangle=(1-\theta \beta) \hat p_t^* +\theta \beta E_t \hat p_{t+1}^\triangle
\end{align}

With the expression of total price being
\begin{align}
&\Rightarrow\hat p_t=(1-\theta) \sum\limits_{k=0}^\infty \theta^k \hat p_{t-k}^\triangle
\\&\Rightarrow \color{red} \hat p_t=(1-\theta)\hat p_t^\triangle +\theta \hat p_{t-1} \text{ (expression of total price based on APSE and PAE)}
\end{align}

From the first equation of total price $\hat p_t=(1-\theta) \sum\limits_{k=0}^\infty \theta^k \hat p_{t-k}^\triangle$ we have
\begin{align}
& \hat p_t =(1-\theta)P_t^\triangle+\theta P_{t-1}
\\& \Rightarrow (1-\theta)P_t^\triangle=\hat p_t - \theta \hat p_{t-1}
\\& \Rightarrow (1-\theta)E_t P_{t+1}^\triangle=E_t \hat p_{t+1} - \theta \hat p_t
\\& \Rightarrow (1-\theta)\beta E_t \hat P_{t+1}^\triangle=\beta E_t \hat p_{t+1} - \theta \beta \hat p_t
\end{align}

From the second equation $\hat p_t=(1-\theta)\hat p_t^\triangle +\theta \hat p_{t-1}$, first use sticky PAE $\hat p_t^\triangle=(1-\theta \beta) \hat p_t^* +\theta \beta E_t \hat p_{t+1}^\triangle$ to substitute
\begin{align}
& \Rightarrow \hat p_t-\theta \hat p_{t-1}=(1-\theta)\hat p_t^\triangle = (1-\theta)(1-\theta \beta)\hat p_t^* 
+(1-\theta)\theta \beta E_t P^\triangle_{t+1}
\end{align}
then use {\color{red} APSE $\hat p_t^*=\hat p_t+\gamma \tilde y_t$} to substitute
\begin{align}
& \Rightarrow \hat p_t -\theta \hat p_{t-1}=(1-\theta)(1-\theta \beta)(\hat p_t+\gamma \tilde y_t)
+(1-\theta)\theta \beta E_t P^\triangle_{t+1}
\end{align}
use the conclusion $(1-\theta)E_t P_{t+1}^\triangle=E_t \hat p_{t+1} - \theta \hat p_t$ above
\begin{align}
& \Rightarrow \hat p_t -\theta \hat p_{t-1}=(1-\theta)(1-\theta \beta)(\hat p_t+\gamma \tilde y_t)+\theta \beta E_t \hat p_{t+1} -\theta^2 \beta \hat p_t
\\& \Rightarrow [1+\theta^2 \beta-(1-\theta)(1-\theta \beta)]\hat p_t -\theta P_{t-1}= (1-\theta)(1-\theta \beta) \gamma \tilde y_t +\theta \beta E_t \text{ (break APSE)}
\\& \Rightarrow 
[1+\theta^2 \beta-(1-\theta)(1-\theta \beta)]\hat p_t -\theta P_{t-1}-\theta \beta \hat p_t = (1-\theta)(1-\theta \beta)\gamma \tilde y_t +\theta \beta (E_t \hat p_{t+1}-\hat p_t)
\\& \Rightarrow 
\theta(\hat p_t -\hat p_{t-1})=(1-\theta)(1-\theta \beta)\gamma \tilde y_t +\theta \beta (E_t \hat P_{t+1}-\hat p_t) 
\text{ (match terms so there is inflation)}
\\& \Rightarrow 
\theta \hat \pi_t = (1-\theta )(1-\theta \beta)\gamma \tilde y_t +\theta \beta E_t \hat \pi_{t+1}
\text{ (here comes inflation)}
\\& \Rightarrow \color{red} \hat \pi_t=k \hat y_t+\beta E_t \hat \pi_{t+1} \text{ (NKPC)}
\end{align}
























\newpage
\subsection{sticky price and sticky wage to labor demand eq, opt problem, with hints}


\begin{itemize}
  \item sticky price
  \begin{itemize}
    \item UMP of household
  \end{itemize}
\end{itemize}

Household face two stages of optimization. Since goods are heterogenous, UMP begins by choosing items $i\in I$. The first stage is sufficient to acquire good demand equation and expression of total price .

We go by method of minimization, as in this case Lagrangian multiplier could be shadow price.
\begin{align}
\begin{split}
& \min \int_{0}^{1} P_{it}C_{it}di
\\& s.t. (\int_{0}^{1}C_{it}^{\frac{\epsilon-1}{\epsilon}}di)^{\frac{\epsilon}{\epsilon-1}} \geqslant C_{t}
\end{split}
\end{align}

Form a Lagrangian, we would have
\begin{align}
&\mathop{\mathscr{L}}\limits_{\{C_{it}\}} =\int_{0}^{1} P_{it}C_{it}di+P_{t}[(\int_{0}^{1}C_{it}^{\frac{\epsilon-1}{\epsilon}}di)^{\frac{\epsilon}{\epsilon-1}}-C_{t}]
\\&\Rightarrow P_{it}=P_{t}\frac{\epsilon}{\epsilon-1}((\int_{0}^{1}C_{it}^{\frac{\epsilon-1}{\epsilon}}di)^{\frac{\epsilon}{\epsilon-1}-1})\frac{\epsilon-1}{\epsilon} C_{it}^{\frac{\epsilon-1}{\epsilon}-1}
\\&\Rightarrow P_{it}=P_{t}[(\int_{0}^{1}C_{it}^{\frac{\epsilon-1}{\epsilon}}di)^{\frac{\epsilon}{\epsilon-1}})^{\frac{1}{\epsilon}}]C_{it}^{\frac{-1}{\epsilon}}
\\&\Rightarrow \frac{P_{it}}{P_{t}}=(\frac{C_{it}}{C_{t}})^{\frac{-1}{\epsilon}}
\\&\Rightarrow \color{red} C_{it}=(\frac{P_{it}}{P_{t}})^{-\epsilon}C_{t} \text{ (the demand curve)}
\end{align}

From $\int_{0}^{1} P_{it}C_{it}di =P_{t}C_{t}$ we have
\begin{align}
&\Rightarrow \int_{0}^{1} P_{it}(\frac{P_{it}}{P_{t}})^{-\epsilon}C_{t} di=P_{t}C_{t}
\\&\Rightarrow \int_{0}^{1} P_{it}(\frac{P_{it}}{P_{t}})^{-\epsilon} di=P_{t}
\\&\Rightarrow \color{red} P_{t}=(\int_{0}^{1}P_{it}^{1-\epsilon}di)^{\frac{1}{1-\epsilon}} \text{ (the expression of total price level)}
\end{align}

\sepline

\begin{itemize}
  \item sticky wage
  \begin{itemize}
    \item PMP
  \end{itemize}
\end{itemize}

With labor market being monopolistic comptitive and sticky wage, we can find labor supply equation and expression of total wage in PMP.

For a representative firm
\begin{equation}
\begin{split}
& \min \int_0^1 W_{jt} N_{ijt}dj
\\& s.t. A_t[(\int_0^1 N_{ijt}^{\frac{\epsilon_W-1}{\epsilon_W}}dj)^{\frac{\epsilon_W}{\epsilon_W-1}}]^{1-\alpha}\geqslant A_t N_{it}^{1-\alpha}
\end{split}
\end{equation}
where $i$ indexing firm, $j$ indexing labor, $t$ is time


Form a Lagrangian
\begin{align}
& \mathcal{L}=\int_0^1 W_{jt} N_{ijt}dj+W_t\{
A_t[(\int_0^1 N_{ijt}^{\frac{\epsilon_w-1}{\epsilon_w}}dj)^{\frac{\epsilon_W}{\epsilon_W-1}}]^{1-\alpha}-A_t N_{it}^{1-\alpha}
\}
\end{align}
where Lagrangian multiplier is total wage level

Using similar techniques to solve the optimization problem results
\begin{align}
& \Rightarrow 
\color{red}
\begin{cases}
N_{ijt}=(\frac{W_{jt}}{W_t})^{-\epsilon_W}N_{it} \text{ (labor supply equation)}
\\
W_t=(\int_0^1 W_{jt}^{1-\epsilon_W}dj)^{\frac{1}{1-\epsilon_W}} \text{ (expression of total wage level)}
\end{cases}
\end{align}

















\newpage
\section{lunshuti, 30'}
\subsection{dynamic inefficiency: solow to ramsey to olg}
\textbf{Solow}

The model has key equations
\begin{equation}
\begin{cases}
  Y=F(K,L)
  \\
  I=S=sY
  \\
  L_{t+1}=(1+n)L_t
  \\
  k_{t+1}=\frac{(1-\delta)k_t+sf(k_t)}{(1+n)}\iff \dot k=sf(k)-(n+\delta)k
\end{cases}
\end{equation}

For a certain production $y=k^a$ at steady state

use $k_{t+1}=\frac{(1-\delta)k_{t}+s f(k_{t})}{1+n}$
\begin{align}
&\Rightarrow(1+n)k^*=(1-\delta)k^*+s f(k^*)
\\&-(n+\delta)k^*=-s f(k^*)
\\&f(k^*)-(n+\delta)k^*=(1-s) f(k^*)=c^*
\\&\Rightarrow c^*=f(k^*)-(n+\delta)k^*
\\&\text{To maximize welfare at steady state} \iff \max c* \iff FOC: k^*_g=(\frac{n+\delta}{\alpha})^{1-\alpha}
\end{align}
also at steady state
\begin{align}
&(1+n)k^*=(1-\delta)k^*+s f(k^*) 
\\&\Rightarrow k^*=(\frac{n+\delta}{s})^{1-\alpha}
\end{align}

for certain $\alpha$ and $s$ it could be
\begin{align}
\Rightarrow k^*<k^*_{g}
\end{align}
meaning there is possible dynamic inefficiency in Solow model.

\textbf{Ramsey}

Ramsey has no dynamic inefficiency, here's why

The model has three key equations
\begin{align}
&\begin{cases}L_{t}=(1+n)L_{t-1} \\
k_{t+1}-(1-\delta)k_t=f(k_t)-c_t \\
\frac{u'(c_{t})}{u'(c_{t+1})}=\beta[f'(k_{t+1})+(1-\delta)]
\end{cases}
\end{align}

Put $k_{t+1}-(1-\delta)k_t=f(k_t)-c_t$ at steady state
\begin{align}
&c^*=f(k^*)-k^*+(1-\delta)k^*=f(k^*)-\delta k^*
\\& \text{To maxmize welfare at steady state}\Rightarrow \text{FOC: } f'(k^{*}_{G})=\delta
\end{align}

Put $\frac{u'(c_{t})}{u'(c_{t+1})}=\beta[f'(k_{t+1})+(1-\delta)]$ at steady state resulting
\begin{align}
&\Rightarrow 1= \beta [f'(k^{*})+(1-\delta)] =\frac{1}{1+\rho} [f'(k^{*})+(1-\delta)] 
\\&\Rightarrow 1+\rho=f'(k^{*})+1- \delta
\\&\Rightarrow f'(k^{*})=\rho +\delta 
\\&\Rightarrow k^{*}_{G}>k^{*}
\end{align}
Capital per capita at steady state is lower than $k_{gold}$, meaning it is dynamic efficient.

\textbf{OLG}

OLG in central planner form:
\begin{align}
\begin{split}
&\max \sum\limits_{t=0}^{\infty}({\beta_{s}})^{t}L_{t}[u(c_{1t})+\beta u(c_{2,t+1})]
\\&s.t. \; 
c_{1t}+\frac{c_{2t}}{1+n}+(1+n)k_{t+1}=f(k_{t})
\end{split}
\end{align}

Equation of motion of capital per capita at the steady states
\begin{align}
&c^{*}\equiv c_{1}^{*}+\frac{c_{2}^{*}}{1+n}=f(k^{*})-(1+n)k^{*}
\end{align}

Since we are maximizing welfare at steady state, let $\frac{\partial  c^{*}}{\partial  k^{*}}=0$ to acquire first order condition
\begin{align}
&\Rightarrow f'(k^{*}_{G})=1+n
\end{align}
using production function $y=k^\alpha$ at equilibrium
\begin{align}
f'(k^{*})&=\alpha (k^{*})^{\alpha-1}
\\&=\alpha\{[\frac{\beta(1-\alpha)}{(1+\beta)(1+n)}]^{\frac{1}{1-\alpha}}\}^{\alpha-1}
\\&=\frac{\alpha}{1-\alpha}\frac{1+\beta}{\beta}(1+n)
\end{align}
for certain value of $\alpha$
\begin{align}
&f'(k^{*})<1+n
\\&\Rightarrow f'(k^{*})<f'(k^{*}_{G})
\\& \Rightarrow k^{*}>k^{*}_{G}
\end{align}
Meaning there can be over-accumulation of capital $\iff$ suffices dynamic inefficiency possible in OLG model.

\textbf{Conclusion}
\begin{itemize}
\tightlist
  \item Due to ad hoc given $s$, Solow model is unable to assure dynamic efficiency.
  \item Due to endogenous saving decision and discount factor $\beta$, Ramsey model is able to achieve dynamic efficiency.
  \item Due to limited life span of individuals, OLG model fails to achieve dynamic efficiency.
\end{itemize}


\newpage
\section{Intermediate macro}

GDP deflator $=\frac{\text{real GDP}}{\text{nominal GDP}}=\frac{\sum P_1 Q_1}{\sum P_0 Q_1}$

CPI $=\frac{\sum P_1 Q_0}{\sum P_0 Q_0}$

Nominal money supply = currency + deposit, $M=C+D$

Basic money = currency + reserve, $B=C+R$

Money multiplier, $\frac{M}{B}=\frac{C+D}{C+R}=\frac{cr+1}{cr+rr}$

quantity equation of money, 
$MV=Py$

Cambridge equation, 
$M=kPY$

Simple equation of money supply, 
$\frac{M}{P}=ky$

Keynesian equation of money supply, 
$\frac{M}{P}=ky-hr$

Friedman equation of money supply, 
$\frac{M}{P}=f(r_b,r_e,r_p,w,y,u)$

Small open economy, 
$NX=EX-IM=S-I$

\includegraphics[width=\textwidth]{images/Small_open.png}

Indirect pricing exchange rate, $e=\varepsilon \frac{P_f}{P}$

sticky price model, $Y=\bar Y +\alpha (P-EP)$


\end{document}